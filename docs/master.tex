\documentclass[a4paper,11pt]{report} 
\usepackage[utf8x]{inputenc}
\usepackage[T1]{fontenc}
\usepackage{lmodern}
\usepackage[polish]{babel}
\author{Marcin Urbański, nr indeksu 237506}
\usepackage{listings}
\usepackage{color}
\usepackage{ucs}
\usepackage{graphicx}

\definecolor{mygreen}{rgb}{0,0.6,0}
\definecolor{mygray}{rgb}{0.5,0.5,0.5}
\definecolor{mymauve}{rgb}{0.58,0,0.82}

\lstset{ 
  backgroundcolor=\color{white},   % choose the background color; you must add \usepackage{color} or \usepackage{xcolor}
  basicstyle=\footnotesize,        % the size of the fonts that are used for the code
  breakatwhitespace=false,         % sets if automatic breaks should only happen at whitespace
  breaklines=true,                 % sets automatic line breaking
  captionpos=b,                    % sets the caption-position to bottom
  commentstyle=\color{mygreen},    % comment style
  deletekeywords={...},            % if you want to delete keywords from the given language
  escapeinside={\%*}{*)},          % if you want to add LaTeX within your code
  extendedchars=true,              % lets you use non-ASCII characters; for 8-bits encodings only, does not work with UTF-8
  frame=single,                    % adds a frame around the code
  keepspaces=true,                 % keeps spaces in text, useful for keeping indentation of code (possibly needs columns=flexible)
  morekeywords={*,...},            % if you want to add more keywords to the set
  numbers=none,                    % where to put the line-numbers; possible values are (none, left, right)
  numbersep=5pt,                   % how far the line-numbers are from the code
  numberstyle=\tiny\color{mygray}, % the style that is used for the line-numbers
  rulecolor=\color{black},         % if not set, the frame-color may be changed on line-breaks within not-black text (e.g. comments (green here))
  showspaces=false,                % show spaces everywhere adding particular underscores; it overrides 'showstringspaces'
  showstringspaces=false,          % underline spaces within strings only
  showtabs=false,                  % show tabs within strings adding particular underscores
  stepnumber=2,                    % the step between two line-numbers. If it's 1, each line will be numbered
  stringstyle=\color{mymauve},     % string literal style
  tabsize=2,                       % sets default tabsize to 2 spaces  
  language=C,
  extendedchars=\true,
  literate={ą}{{\k{a}}}1
             {Ą}{{\k{A}}}1
             {ę}{{\k{e}}}1
             {Ę}{{\k{E}}}1
             {ó}{{\'o}}1
             {Ó}{{\'O}}1
             {ś}{{\'s}}1
             {Ś}{{\'S}}1
             {ł}{{\l{}}}1
             {Ł}{{\L{}}}1
             {ż}{{\.z}}1
             {Ż}{{\.Z}}1
             {ź}{{\'z}}1
             {Ź}{{\'Z}}1
             {ć}{{\'c}}1
             {Ć}{{\'C}}1
             {ń}{{\'n}}1
             {Ń}{{\'N}}1
}


\begin{document}
\begin{titlepage}
\begin{center}
\begin{Huge}
	\emph{Projekt z przedmiotu:}
	\\
	\textbf{Techniki Kompilacji}
\end{Huge}
\begin{normalsize}
	\\.\\.\\.\\.\\.\\.\\.\\.\\.\\.\\.\\.\\.
	\\.\\.\\.\\.\\.\\.\\.\\.\\.\\.\\.\\.\\
	autor: Marcin Urbański
	\\
	nr albumu: 237506
	\\
	opiekun: dr inż. Krzysztof Cabaj
\end{normalsize}		
\end{center}
\end{titlepage}


\chapter{Wstęp}

\section{Treść zadania 5.}

Celem zadania jest sparsowanie plików znajdujących się w podanym katalogu (pliki tekstowe i
binarne) i wykrycie tych, które zawierają poprawne żądania protokołu HTTP. 

Dla wykrytych sesji HTTP
należy odnaleźć plik z odpowiedzią (nazwy plików zawierają adresy i porty źródłowe i docelowy,
odpowiedź ma je w odwrotnej kolejności) i zapisać przesłane dane (bez nagłówków odpowiedzi
HTTP) w oddzielnym pliku. 

Dodatkowo program powinien tworzyć plik statystyki i dla każdej wykrytej
sesji zapisywać w jednej linii istotne informacje: 

\begin{itemize}
\item
	nazwę pliku z żądaniem HTTP
\item
	metodę
\item
    URL
\item
    wielkość pliku
\item
    nazwę pliku z przesyłaną treścią
\end{itemize}
	
\section{Interpretacja zadania}

W ramach projektu zostanie wykonany program, którego zadaniem jest sprawdzenie czy pliku z zadanego katalogu zawierają poprawne żądania i odpowiedzi HTTP. Prezentowany poniżej program jest parserem, którego zadaniem jest również gromadzenie informacji o przeszukiwanych plikach i znalezionych sesjach. Pliki służące do testowania programu zostaną wygenerowane za pomocą programu tcpflow. Zadanie zostanie wykonane obiektowo w języku C++. Stworzone zostaną klasy, odpowiadające opisanym niżej modułom.

\section{Dodatkowe biblioteki}

Zamierzam skorzystać jedynie z biblioteki boost, w celu pobrania listy plików z danego katalogu.

\section{Obsługiwane systemy}

Program powinien działać zarówno pod Linuxem, jak i Windowsem
\\ Testowane na:

\section{Przykładowe poprawne żądanie HTTP}

\begin{lstlisting}

GET /css/all.css HTTP/1.1
Host: www.abc.org
Connection: keep-alive
Cache-Control: max-age=0
Accept: text/css,*/*;q=0.1
If-None-Match: "2f759-38a-49d87fa295080"
If-Modified-Since: Wed, 02 Mar 2011 23:07:30 GMT
User-Agent: Mozilla/5.0 (X11; Linux x86_64) AppleWebKit/537.36 (KHTML, like Gecko) Chrome/33.0.1750.152 Safari/537.36
DNT: 1
Referer: http://www.abc.org/
Accept-Encoding: gzip,deflate,sdch
Accept-Language: en-US,en;q=0.8,pl;q=0.6
Cookie: __utma=260800520.62948572.1395226092.1395248356.1395597204.3; __utmb=260800520.1.10.1395597204; __utmc=260800520; __utmz=260800520.1395597204.3.3.utmcsr=google|utmccn=(organic)|utmcmd=organic|utmctr=(not%20provided)

.
\end{lstlisting}

\section{Przykładowa poprawna odpowiedź HTTP}

\begin{lstlisting}

HTTP/1.1 200 OK
Content-Type: text/css
Timing-Allow-Origin: *
Expires: Sun, 23 Mar 2014 17:53:43 GMT
Date: Sun, 23 Mar 2014 17:53:43 GMT
Cache-Control: private, max-age=86400
Content-Length: 476
X-Content-Type-Options: nosniff
X-Frame-Options: SAMEORIGIN
X-XSS-Protection: 1; mode=block
Server: GSE
Alternate-Protocol: 80:quic

@font-face {
  font-family: 'Ubuntu';
  font-style: normal;
  font-weight: 400;
  src: local('Ubuntu'), url(http://themes.googleusercontent.com/static/fonts/ubuntu/v5/_xyN3apAT_yRRDeqB3sPRg.woff) format('woff');
}
@font-face {
  font-family: 'Ubuntu';
  font-style: normal;
  font-weight: 700;
  src: local('Ubuntu Bold'), local('Ubuntu-Bold'), url(http://themes.googleusercontent.com/static/fonts/ubuntu/v5/0ihfXUL2emPh0ROJezvraD8E0i7KZn-EPnyo3HZu7kw.woff) format('woff');
}

\end{lstlisting}

\chapter{Moduły}

\section{Wczytywanie plików wraz z analizą treści}

Moduł odpowiedzialny za sprawdzenie plików w danym katalogu. Wywołuje on również moduły odpowiedzialne za sprawdzenie poprawności żądań i odpowiedzi HTTP.

\subsection{Pseudokod funkcji checkDirectory()}

\begin{lstlisting}

 void checkDirectory(String directory) {
   // raportuj rozpoczęcie skanowania
   Log.report (SCAN_RESULT, STARTED);
   // lista plików w zadanym katalogu
   String files[] = getFilesFromDirectory(directory);
   // pętla skanująca dany katalog
   for(int i=0; i<files.size(); i++) {
     // jeśli plik nie ma poprawnej nazwy
     // przejdź do kolejnego
     if (isRequestFileNameIncorrect(files[i])) {
       continue;
     }   
     // uzyskaj deskryptor pliku 
     File requestFile = new File(file[i]);  
     // rozpocznij sesję z danym plikiem
     Log.start (requestFile);
     // sprawdź, czy zadany plik zawiera żądanie HTTP
     int stateRequest = Request(requestFile);
     // raportuj rezultat sprawdzania żądania HTTP
     Log.report (REQUEST_RESULT, stateRequest);
     // zwróć deskryptor pliku z odpowiedzią
     File responseFile = \
         getResponseFileFromRequestFile(requestFile);
     // sprawdź, czy zadany plik zawiera odpowiedź HTTP
     int stateResponse = Response(responseFile);
     // raportuj rezultat sprawdzania odpowiedzi HTTP
     Log.report (RESPONSE_RESULT, stateResponse);
     // zakończ sesję dla danego pliku
     Log.finish ();
   }
   // raportuj stan skanowania
   Log.report (SCAN_RESULT, FINISHED);
 }

\end{lstlisting}

\subsection{Poprawna nazwa pliku}

\begin{lstlisting}

 file_name = source "-" destination
 
 source = source_address "." source_port
 source_address = 3DIGIT "." 3DIGIT "." 3DIGIT "." 3DIGIT
 source_port = 5DIGIT
 
 destination = destination_address "." destination_port
 destination_address = 3DIGIT "." 3DIGIT "." 3DIGIT "." 3DIGIT
 destination_port = 5DIGIT

\end{lstlisting}

\subsection{Przykładowe nazwy plików}

\begin{lstlisting}

 023.061.248.090.00443-192.168.000.011.54796
 023.061.248.096.00443-192.168.000.011.56132
 // request
 192.168.000.011.39677-098.136.223.068.00080
 // response
 098.136.223.068.00080-192.168.000.011.39677

\end{lstlisting}

\subsection{Pseudokod funkcji isRequestFileNameIncorrect()}

Zwraca true, jeśli dany plik ma niepoprawną nazwę

\begin{lstlisting}

 bool isRequestFileNameIncorrect (File filename) {
   // indeks znaku rozdzielającego
   int indexOfSeparator = getPositionOf(filename, '-');
   // jeśli nie znaleziono znaku rozdzielającego
   if (indexOfSeparator == NOT_FOUND) {
     return true;
   }
   // sprawdzaj port i adres źródłowy
   if (source (filename.substring(0, indexOfSeparator)) {
     // sprawdź port i adres docelowy
     if (destination (filename.substr (indexOfSeparator+1)) {
       // zwróć informację o poprawnej nazwie pliku
       return false;
     }
   }
   // plik niepoprawny
   return true;
 }

\end{lstlisting}

\begin{lstlisting}

 bool source (String source) {
   // sprawdź adres i port źródłowy
   if (checkAddressAndPort (source)) {
     return true;
   } else {
     return false;
   }
 }
 
 bool destination (String destination) {
   // sprawdź adres i port docelowy
   if (checkAddressAndPort (destination)) {
     return true;
   } else {
     return false;
   }
 }
\end{lstlisting}

\begin{lstlisting}

 bool checkAddressAndPort (String text) {  
   // sprawdź poprawność adresu
   for (int i=0; i < 4; i++) {
     // pierwszy segment adresu
     int address_segment = atoi(text.substr(i*4, i*4+3);
     // sprawdź czy mieści się w zakresie
     if (address_segment > 255 || address_segment < 0) {
       return false;
     }
     // sprawdź czy są wymagne kropki
     if (text [i*4+3] != '.') {
       return false;
     } 
   }
   // sprawdź poprawność portu
   int port = atoi(text.substr(i*4);
   // sprawdź zakres portu
   if (port > 65535 || port < 0) {
       return false;
   }
   // zwróć informacje o sukcesie
   return true;
 }

\end{lstlisting}

\section{Sprawdzenie poprawności żądań HTTP}

Zadaniem tego modułu jest analiza treści pliku w celu znalezienia poprawnych żądań.

\subsubsection{Gramatyka żądania HTTP}

\begin{lstlisting}

 Request        = Request-Line [request-header] CRLF
 Request-Line   = Method SP Request-URI SP HTTP-Version CRLF 
 request-header = *(field-name ":" [ field-value ] | Host) CRLF
 field-name     = token
 Method         = "OPTIONS" 
                 |"GET"
                 |"HEAD"
                 |"POST" 
                 |"PUT"
                 |"DELETE" 
                 |"TRACE" 
                 |"CONNECT"
 Request-URI    = "*" | absoluteURI | abs_path
 HTTP-Version   = "HTTP" "/" 1*DIGIT "." 1*DIGIT
 Host           = "Host" ":" host [ ":" port ] 
 absoluteURI    = "http://" host [abs_path] [":" port]
 field-value    = VALUE_TEXT
 token          = TOKEN_TEXT
 host           = HOST_TEXT
 abs_path       = PATH_TEXT
 port           = 1*DIGIT
 SP             = " "
 CRLF           = "\n"
 DIGIT          = "0"|"1"|"2"|"3"|"4"|"5"|"6"|"7"|"8"|"9"
 HOST_TEXT    = "A"|"B"|"C"|"D"|"E"|"F"|"G"|"H"|"I"|"J"|"K"|"L"|
               |"M"|"N"|"O"|"P"|"Q"|"R"|"S"|"T"|"U"|"V"|"W"|"X"|
               |"Y"|"Z"|"a"|"b"|"c"|"d"|"e"|"f"|"g"|"h"|"i"|"j"|
               |"k"|"l"|"m"|"n"|"o"|"p"|"q"|"r"|"s"|"t"|"u"|"v"|
               |"w"|"x"|"y"|"z"|"-"|"."| DIGIT
 PATH_TEXT    = HOST_TEXT |"_"|"~"|":"|"/"|"?"|"#"|"["|"]"|
               |"@"|"!"|"$"|"&"|"'"|"("|")"|"*"|"+"|","|";"|"="
 VALUE_TEXT   = *(TOKEN_TEXT | ":")
 TOKEN_TEXT   = *<all characters excluding CR LF ":">

\end{lstlisting}

\subsection{Request}

\begin{lstlisting}

 Request = Request-Line [header] CRLF
 
\end{lstlisting}

\subsubsection{Pseudokod funkcji Request()}

\begin{lstlisting}

 void Request(File file) {
   
   // kursor odpowiada pozycji sprawdzanego znaku w file
   long request_cursor = 0;
   // dlugość pliku
   long file_length = file.getLength();
   // petla sprawdzajaca czy w pliku znajduje się Request-Line
   while (request_cursor < file_length) {
   
     // sprawdź czy od pozycji request_cursor
     // może rozpoczynać się Request line
     // jeśli Request-Line() znajdzie Request-Line
     //   request-line_cursor odpowiada pozycji w pliku za
     //   Request-Line 
     // jeśli nie znajdzie
     //   jest równa NOT_FOUND   
     long request-line_cursor = /
                 Request-Line (request_cursor, file);
         
     // Wykryto Request-Line
     // sprawdź czy w pliku znajdują się pola nagłówka
     if (request-line_cursor != NOT_FOUND) {
             
       // header jest opcjonalny - jeśli nie jest obecny
       //   to header_cursor równe jest request-line_cursor
       // w przeciwnym razie odpowiada pozycji za header'em
       header_cursor = header(request-line_cursor, file);
             
       // zakończ pętlę
       break;
       
     }
     // przesuń kursor, ponieważ nie znaleziono żądania
     request_cursor++; 
   }
 }
 
\end{lstlisting}

\subsection{Request-Line}

\begin{lstlisting}

 Request-Line = Method SP Request-URI SP HTTP-Version CRLF

\end{lstlisting}

\subsubsection{Pseudokod funkcji Request-Line()}

\begin{lstlisting}

 int Request-Line (long request-line_cursor, File file) {
   // sprawdź czy znaleziono metodę
   long method_cursor = Method (request-line_cursor, file);
   // zwró
   if (method_cursor == NOT_FOUND) {
      // jeśli nie znaleziono Method zwróć niepowodzenie
      return NOT_FOUND;
   }
   // sprawdź czy za Method znajduje się znak SP
   char ch = file.getChar(method_cursor);
   if (ch == SP) {
     method_cursor++;
     long request-uri_cursor = Request-URI (method_cursor, file)
     if (request-uri_cursor == NOT_FOUND) {
       // jeśli nie znaleziono Method zwróć niepowodzenie
       return NOT_FOUND;
     }
   } else {
     // jeśli nie znaleziono SP zwróć niepowodzenie
     return NOT_FOUND;
   }
   
   // sprawdź czy za Request-URI jest SP
   ch = file.getChar(request-uri_cursor);
   
   if (ch == SP) {
     // sprawdzamy poprawność HTTP-Version
     request-uri_cursor++;
     long http-version_cursor = /
         HTTP-Version (request-uri_cursor, file);
     // jeśli nie znaleziono zwróć niepowodzenie     
     if (http-version_cursor == NOT_FOUND) {
       return NOT_FOUND;
     }
   } else {
     // nie znaleziono znaku SP - niepowodzenie
     return NOT_FOUND;   
   }
   // sprawdź czy za HTTP-VERSION jest CRLF
   ch = file.getChar(http_version_cursor);
   if (ch == CRLF) {
     // Sprawdzanie Request-Line zakończone powodzeniem
     return ++http_version_cursor; 
   } else {
     // nie znaleziono znaku CRLF - niepowodzenie
     return NOT_FOUND;   
   }
 }

\end{lstlisting}

\subsection{Method}

\begin{lstlisting}

 Method = "OPTIONS" 
        | "GET"
        | "HEAD"
        | "POST"
        | "PUT"
        | "DELETE"
        | "TRACE"
        | "CONNECT"

\end{lstlisting}

\subsubsection{Pseudokod funkcji Method()}

\begin{lstlisting}

 long Method (long method_cursor, File file) {
   
   // sprawdź od pozycji method_cursor w pliku file
   // rozpoczynają się tablice znaków z Method_array
   // w przypadku niepowodzenia zwróć NOT_FOUND
   // w przypadku powodzenia zwróć pozycję za znalezioną metodą 
   
   // zaraportuj rezultat
   Log.report(METHOD, method);
 
 }

\end{lstlisting}

\subsection{Request-URI}

\begin{lstlisting}

 Request-URI = "*" | absoluteURI | abs_path

\end{lstlisting}

\subsubsection{Pseudokod funkcji Request-URI()}

\begin{lstlisting}

 long Request-URI (long request_uri_cursor, File file) {
   char ch = file.getChar(request_uri_cursor);
   // wybierz odpowiednią opcję na podstawie pierwszego znaku   
   switch (ch) {
     case '*':
       return ++request_uri_cursor;
     case 'h':
       long absolute_uri_cursor = /
           absoluteURI(request_uri_cursor, file);
       // nie znaleziono absoluteURI
       if (absolute_uri_cursor == NOT_FOUND) {
         return NOT_FOUND
       } else {
         // zwróć poprawny wynik
         return absolute_uri_cursor;       
       }
     case '/':
       long abs_path = abs_path(request_uri_cursor, file);
       nie znaleziono abs_path
       if (abs_path_cursor == NOT_FOUND) {
         return NOT_FOUND
       } else {
         // zwróć poprawny wynik
         return abs_path_cursor;       
       }
     default:
       return NOT_FOUND;
   }  
 }

\end{lstlisting}

\subsection{absoluteURI}

\begin{lstlisting}

 absoluteURI = "http://" host [abs_path] [":" port]

\end{lstlisting}

\subsubsection{Pseudokod funkcji absoluteURI()}

\begin{lstlisting}

 long absoluteURI (long absolute_uri_cursor, File file) {
   // sprawdź czy w pliku jest tekst "http://"
   String http = file.getString (absolute_uri_cursor, 7);
   // jeśli nie znaleziono zwróć informację
   if (!http.equals("http://") {
     return NOT_FOUND;
   }
   // przesuń kursor poza znaleziony tekst
   absolute_uri_cursor += 7;
   // sprawdź obecność host
   long host_cursor = host (absolute_uri_cursor, file);
   // jeśli nie znaleziono zwróć informację
   if (host == NOT_FOUND) {
     return NOT_FOUND;
   }   
   // sprawdź obecność abs_path, jest to pole opcjonalne
   long abs_path_cursor = abs_path (host_cursor, file);
   // jeśli za zwróconym kursorem jest ":" sprawdź port
   char ch = getChar (abs_path_cursor, file);
   if (ch == ':') {
     ++abs_path_cursor;
     long port_cursor = port (abs_path_cursor, file);
     return port_cursor;
   }
   // zwróć nową pozycję kursora
   return host_cursor;   
 }

\end{lstlisting}

\subsection{host}

\begin{lstlisting}

 host         = 1*HOST_TEXT
 HOST_TEXT    = "A"|"B"|"C"|"D"|"E"|"F"|"G"|"H"|"I"|"J"|"K"|"L"|
               |"M"|"N"|"O"|"P"|"Q"|"R"|"S"|"T"|"U"|"V"|"W"|"X"|
               |"Y"|"Z"|"a"|"b"|"c"|"d"|"e"|"f"|"g"|"h"|"i"|"j"|
               |"k"|"l"|"m"|"n"|"o"|"p"|"q"|"r"|"s"|"t"|"u"|"v"|
               |"w"|"x"|"y"|"z"|"0"|"1"|"2"|"3"|"4"|"5"|"6"|"7"|
               |"8"|"9"|"-"|"."

\end{lstlisting}

\subsubsection{Pseudokod funkcji host}

\begin{lstlisting}

 long host (long host_cursor, File file) {
 
   // pobieraj znaki aż napotkasz " ", ":" lub "/"
   //   zwróć kursor za sprawdzanonym tekstem
   // jeśli znak nie należy do HOST_TEXT zwróć NOT_FOUND
   
   // zaraportuj dane
   Log.report (HOST, host);
   
 }

\end{lstlisting}

\subsection{port}

\begin{lstlisting}

 port  = 1*DIGIT
 DIGIT = "0"|"1"|"2"|"3"|"4"|"5"|"6"|"7"|"8"|"9"

\end{lstlisting}

\subsubsection{Pseudokod funkcji port}

\begin{lstlisting}

 long port (long port_cursor, File file) {
 
   // pobieraj kolejne znaki póki są cyframi
   //   koniecznie co najmniej jedna cyfra
   // jeśli stworzona liczba zmieści się w przedziale
   //   zwróć kursor za ostatnią liczbą
   // w każdym innym przypadku NOT_FOUND
      
 }

\end{lstlisting}

\subsection{abs path}

\begin{lstlisting}

 abs_path     = 1*PATH_TEXT
 PATH_TEXT    = "A"|"B"|"C"|"D"|"E"|"F"|"G"|"H"|"I"|"J"|"K"|"L"|
               |"M"|"N"|"O"|"P"|"Q"|"R"|"S"|"T"|"U"|"V"|"W"|"X"|
               |"Y"|"Z"|"a"|"b"|"c"|"d"|"e"|"f"|"g"|"h"|"i"|"j"|
               |"k"|"l"|"m"|"n"|"o"|"p"|"q"|"r"|"s"|"t"|"u"|"v"|
               |"w"|"x"|"y"|"z"|"0"|"1"|"2"|"3"|"4"|"5"|"6"|"7"|
               |"8"|"9"|"-"|"."|"_"|"~"|":"|"/"|"?"|"#"|"["|"]"|
               |"@"|"!"|"$"|"&"|"'"|"("|")"|"*"|"+"|","|";"|"="

\end{lstlisting}

\subsubsection{Pseudokod funkcji abs path()}

\begin{lstlisting}

 long abs_path (long abs_path_cursor, File file) {
 
   char ch = getChar (abs_path_cursor, file);
   
   // pobieraj znaki aż napotkasz " " lub ":"
   //   zwróć kursor za sprawdzonym tekstem
   // jeśli znak nie należy do PATH_TEXT zwróć NOT_FOUND
     
 }

\end{lstlisting}

\subsection{HTTP-Version}

\begin{lstlisting}

 HTTP-Version = "HTTP" "/" 1*DIGIT "." 1*DIGIT

\end{lstlisting}

\subsubsection{Pseudokod funkcji HTTP-Version()}

\begin{lstlisting}

 long HTTP-Version (long http_version_cursor, File file) {
     // sprawdź czy od pozycji kursora znaki w pliku
     //   są równe "HTTP/"
     // pobieraj znak po znaku cyfry aż do spotkania "."
     //   jeśli spotkasz inny znak niż cyfrę - zwróć NOT_FOUND
     //   musi być co najmniej jedna cyfra
     // pobieraj znak po znaku cyfry aż do spotkania CRLF
     //   analogicznie jak wcześniej
 }

\end{lstlisting}

\subsection{request-header}

\begin{lstlisting}

 request-header = *(field-name ":" [field-value] | Host) CRLF
 field-name     = token
 field-value    = VALUE_TEXT
 token          = TOKEN_TEXT

\end{lstlisting}

\subsubsection{Pseudokod funkcji request-header()}

\begin{lstlisting}

 long request-header(long request-header_cursor, File file) {
   // (*) pobieraj znaki aż do spotkania ":"
   // utwórz z nich String field-name
   //   jeśli field-name równe "Host" rozpatrz ten przypadek
   //   jeśli nie pobieraj znaki za ":", aż do spotkania CRLF
   //     jeśli kolejny to znów CRLF
   //       zwróć pozycję nowego kursora
   //     jeśli to inny znak to wróć do (*)
   // w przypadku niepowodzenia zwróć niezmienioną pozycję
   // request_header_cursor
 }

\end{lstlisting}

\subsection{Host}

\begin{lstlisting}

 Host = "Host" ":" host [ ":" port ] 

\end{lstlisting}

\subsubsection{Pseudokod funkcji Host()}

\begin{lstlisting}

 long Host(long host_cursor, File file) {
 
   // sprawdź, czy za "Host:" znajduje się host
   // sprawdź, czy za hostem znajduje się port - pole opcjonalne
 } 

\end{lstlisting}

\section{Sprawdzenie poprawności odpowiedzi HTTP}

Zadaniem tego modułu jest analiza treści pliku w celu znalezienia poprawnych odpowiedzi.

\subsubsection{Gramatyka odpowiedzi HTTP}

\begin{lstlisting}

 Response        = Status-Line [response-header] 
                   CRLF [message-body]
 Status-Line     = HTTP-Version SP Status-Code SP 
                   Reason-Phrase CRLF 
 response-header = (field-name ":" [field-value] 
                 | Content-Length) CRLF
 field-name      = token
 field-value     = VALUE_TEXT
 token           = 1*(TOKEN_TEXT)
 Status-Code     = "100"  ; Continue
                 | "101"  ; Switching Protocols
                 | "200"  ; OK
                 | "201"  ; Created
                 | "202"  ; Accepted
                 | "203"  ; Non-Authoritative Information
                 | "204"  ; No Content
                 | "205"  ; Reset Content
                 | "206"  ; Partial Content
                 | "300"  ; Multiple Choices
                 | "302"  ; Found
                 | "303"  ; See Other
                 | "304"  ; Not Modified
                 | "305"  ; Use Proxy
                 | "307"  ; Temporary Redirect
                 | "400"  ; Bad Request
                 | "401"  ; Unauthorized
                 | "402"  ; Payment Required
                 | "403"  ; Forbidden
                 | "404"  ; Not Found
                 | "405"  ; Method Not Allowed
                 | "406"  ; Not Acceptable
                 | "407"  ; Proxy Authentication Required
                 | "408"  ; Request Time-out
                 | "409"  ; Conflict
                 | "410"  ; Gone
                 | "411"  ; Length Required
                 | "412"  ; Precondition Failed
                 | "413"  ; Request Entity Too Large
                 | "414"  ; Request-URI Too Large
                 | "415"  ; Unsupported Media Type
                 | "416"  ; Requested range not satisfiable
                 | "417"  ; Expectation Failed
                 | "500"  ; Internal Server Error
                 | "501"  ; Not Implemented
                 | "502"  ; Bad Gateway
                 | "503"  ; Service Unavailable
                 | "504"  ; Gateway Time-out
                 | "505"  ; HTTP Version not supported

 HTTP-Version    = "HTTP" "/" 1*DIGIT "." 1*DIGIT
 Reason-Phrase   = 1*VALUE_TEXT
 Content-Length  = "Content-Length" ":" 1*DIGIT 
 SP              = " "
 CRLF            = "\n"
 DIGIT           = "0"|"1"|"2"|"3"|"4"|"5"|"6"|"7"|"8"|"9"
 TOKEN_TEXT      = 1*(<all characters excluding CR LF ":">)
 VALUE_TEXT      = 1*(TOKEN_TEXT | ":")

\end{lstlisting}

\subsection{Response}

\begin{lstlisting}

 Response = Status-Line [response-header] CRLF [message-body]
 
\end{lstlisting}

\subsubsection{Pseudokod funkcji Response()}

\begin{lstlisting}

 void Response(File file) {
   
   // kursor odpowiada pozycji sprawdzanego znaku w file
   long response_cursor = 0;
   // dlugość pliku
   long file_length = file.getLength();
   // petla sprawdzajaca czy w pliku znajduje się Response-Line
   while (response_cursor < file_length) {
   
     // sprawdź czy od pozycji response_cursor
     // może rozpoczynać się Status-Line
     // jeśli Status-Line() znajdzie Status-Line
     //   status_line_cursor odpowiada pozycji w pliku za
     //   Status-Line 
     // jeśli nie znajdzie
     //   jest równa NOT_FOUND   
     long status_line_cursor = /
                 Status-Line (request_cursor, file);
         
     // Wykryto Status-Line
     // sprawdź czy w pliku znajdują się pola nagłówka
     if (status_line_cursor != NOT_FOUND) {
             
       // header jest opcjonalny - jeśli nie jest obecny
       //   to header_cursor równe jest status_line_cursor
       // w przeciwnym razie odpowiada pozycji za header'em
       header_cursor = response_header(status_line_cursor, file);
             
       // sprawdź message-body i stwórz plik z danymi        
       message-body(header_cursor, file);
       // zakończ pętlę
       break;
       
     }
     // przesuń kursor, ponieważ nie znaleziono żądania
     response_cursor++; 
   }
 }
 
\end{lstlisting}

\subsection{Status-Line}

\begin{lstlisting}

 Status-Line = HTTP-Version SP Status-Code SP Reason-Phrase CRLF

\end{lstlisting}

\subsubsection{Pseudokod funkcji Status-Line()}

\begin{lstlisting}

 long Status-Line (long status_line_cursor, File file) {
   // sprawdź czy znaleziono metodę
   long http_version__cursor = HTTP-Version (status_line_cursor, file);
   // zwróć niepowodzenie w razie nie znalezienia
   if (http_version_cursor == NOT_FOUND) {
      // jeśli nie znaleziono Method zwróć niepowodzenie
      return NOT_FOUND;
   }
   // sprawdź czy za Method znajduje się znak SP
   char ch = file.getChar(http_version_cursor);
   if (ch == SP) {
     http_version_cursor++;
     long status_code_cursor = Status-Code (http_version_cursor, file)
     if (status_code_cursor == NOT_FOUND) {
       // jeśli nie znaleziono Status-Code zwróć niepowodzenie
       return NOT_FOUND;
     }
   } else {
     // jeśli nie znaleziono SP zwróć niepowodzenie
     return NOT_FOUND;
   }
   
   // sprawdź czy za Status-Code jest SP
   ch = file.getChar(status_code_cursor);
   
   if (ch == SP) {
     // sprawdzamy poprawność Reason-Phrase
     status_code_cursor++;
     long reason_phrase_cursor = /
         Reason-Phrase (status_code_cursor, file);
     // jeśli nie znaleziono zwróć niepowodzenie     
     if (reason_phrase_cursor == NOT_FOUND) {
       return NOT_FOUND;
     }
   } else {
     // nie znaleziono znaku SP - niepowodzenie
     return NOT_FOUND;   
   }
   // sprawdź czy za Reason-Phrase jest CRLF
   ch = file.getChar(reason_phrase_cursor);
   if (ch == CRLF) {
     // Sprawdzanie Reason-Phrase zakończone powodzeniem
     return ++reason_phrase_cursor; 
   } else {
     // nie znaleziono znaku CRLF - niepowodzenie
     return NOT_FOUND;   
   }
 }
 
\end{lstlisting}

\subsection{HTTP-Version}

\begin{lstlisting}

 HTTP-Version = "HTTP" "/" 1*DIGIT "." 1*DIGIT

\end{lstlisting}

\subsubsection{Pseudokod funkcji HTTP-Version()}

\begin{lstlisting}

 long HTTP-Version (long http_version_cursor, File file) {
     
     // sprawdź czy od pozycji kursora znaki w pliku
     //   są równe "HTTP/"
     // pobieraj znak po znaku cyfry aż do spotkania "."
     //   jeśli spotkasz inny znak niż cyfrę - zwróć NOT_FOUND
     //   musi być co najmniej jedna cyfra
     // pobieraj znak po znaku cyfry aż do spotkania SP
     //   analogicznie jak wcześniej
 }

\end{lstlisting}

\subsection{Status-Code}

\begin{lstlisting}

 Status-Code = "100"  ; Continue
             | "101"  ; Switching Protocols
             | "200"  ; OK
             | "201"  ; Created
             | "202"  ; Accepted
             | "203"  ; Non-Authoritative Information
             | "204"  ; No Content
             | "205"  ; Reset Content
             | "206"  ; Partial Content
             | "300"  ; Multiple Choices
             | "301"  ; Moved Permanently
             | "302"  ; Found
             | "303"  ; See Other
             | "304"  ; Not Modified
             | "305"  ; Use Proxy
             | "307"  ; Temporary Redirect
             | "400"  ; Bad Request
             | "401"  ; Unauthorized
             | "402"  ; Payment Required
             | "403"  ; Forbidden
             | "404"  ; Not Found
             | "405"  ; Method Not Allowed
             | "406"  ; Not Acceptable
             | "407"  ; Proxy Authentication Required
             | "408"  ; Request Time-out
             | "409"  ; Conflict
             | "410"  ; Gone
             | "411"  ; Length Required
             | "412"  ; Precondition Failed
             | "413"  ; Request Entity Too Large
             | "414"  ; Request-URI Too Large
             | "415"  ; Unsupported Media Type
             | "416"  ; Requested range not satisfiable
             | "417"  ; Expectation Failed
             | "500"  ; Internal Server Error
             | "501"  ; Not Implemented
             | "502"  ; Bad Gateway
             | "503"  ; Service Unavailable
             | "504"  ; Gateway Time-out
             | "505"  ; HTTP Version not supported

\end{lstlisting}

\subsubsection{Pseudokod funkcji Status-Code()}

\begin{lstlisting}

 long Status-Code (long method_cursor, File file) {
   
   // pobierz trzy znaki od pozycji kursora
   String code_text = file.getString(method_cursor, 3);
   // przekształć znaki na liczbę
   int code = atoi(code_text);
   // jeśli code_text nie trzech cyfr zwróć niepowodzenie   
   if (code == -1) {
     return NOT_FOUND;
   } else {
     // sprawdź czy code znajduje się w tablicy Status-Code
     for (int i=0; i<Status-Code_Array.size(); i++) {
       if (code == Status-Code_Array[i]) {
          return method_cursor+3;
       }
     }
   }  
   return NOT_FOUND;
 }

\end{lstlisting}

\subsection{Reason-Phrase}

\begin{lstlisting}

 Reason-Phrase   = *TEXT 
 TEXT            = <all characters excluding CR, LF>

\end{lstlisting}

\subsubsection{Pseudokod funkcji Reason-Phrase()}

\begin{lstlisting}

 long Reason-Phrase (long request_uri_cursor, File file) {
   // pobieraj znaki z file począwszy od request_uri_cursor
   //   aż do napotkania CR LF
   // zwróć pozycję za ostatnim pobranym znakiem
 }

\end{lstlisting}

\subsection{response-header}

\begin{lstlisting}

 response-header = (field-name ":" [field-value] 
                 | Content-Length) CRLF
 field-name      = token
 field-value     = VALUE_TEXT
 token           = 1*(TOKEN_TEXT)
 TOKEN_TEXT      = 1*(<all characters excluding CR LF ":">)
 VALUE_TEXT      = 1*(TOKEN_TEXT | ":")

\end{lstlisting}

\subsubsection{Pseudokod funkcji response-header()}

\begin{lstlisting}

 long response-header(long response_header_cursor, File file) {
   // (*) pobieraj znaki aż do spotkania ":"
   // utwórz z nich String field_name
   //   jeśli field-name równe "Content_Length" rozpatrz ten przypadek
   //   jeśli nie pobieraj znaki za ":", aż do spotkania CRLF
   //     jeśli kolejny to znów CRLF
   //       zwróć pozycję nowego kursora
   //     jeśli to inny znak to wróć do (*)
   // zwróć pozycję nowego kursora
   // w przypadku niepowodzenia zwróć niezmienioną pozycję
   // response_header_cursor
 }

\end{lstlisting}

\subsection{Content-Length}

\begin{lstlisting}

 Content-Length = "Content-Length" ":" 1*DIGIT

\end{lstlisting}

\subsubsection{Pseudokod funkcji Content-Length()}

\begin{lstlisting}

 long Content-Length (long absolute_uri_cursor, File file) {
 
   // sprawdź, czy za "Content-Length:" znajduje się ciąg cyfr
   //   przy czym musi być to conajmniej jedna cyfra
   //   pobieraj cyfry aż do spotkania znaku CRLF
   //   w przypadku znaku spoza zakresu zwróć niepowodzenie
   
   // zaraportuj dane
   Log.report (CONTENT_LENGTH, content_length); 
 }

\end{lstlisting}

\subsection{message-body}

\begin{lstlisting}

 message-body  = *TEXT

\end{lstlisting}

\subsubsection{Pseudokod funkcji message-body()}

\begin{lstlisting}

 long message-body(long message_body_cursor, File file) {
 
   // utwórz tablicę bajtów 
   //   od message_body_cursor aż do końca pliku
   // utwórz z tej tablicy plik np. wysyłając tablicę 
   //   do funkcji logującej, która będzie to obsługiwała
   Log.report (CREATE_FILE, message_body_array);
 } 

\end{lstlisting}

\section{Logowanie poprawnych sesji i tworzenie statystyk}

\subsection{Klasa Log}

Klasa, której statyczne metody będą zapisywać dane służące do tworzenia statystyk oraz informowania użytkownika o wszelkiego rodzaju błędach i etapach działania programu.

\subsection{Funkcja start()}

Rozpoczyna sesję dla danego pliku

\subsection{Funkcja finish()}

Kończy sesję dla danego pliku

\subsection{Funkcja report()}

Zaraportuj zdarzenie dla sesji pliku lub skanowania
Działa i wypisuje dane na konsole lub do pliku, w zależności od określonych przez użytkownika parametrów.
Dzięki niej raportowane są zdarzenia takie jak:

\begin{itemize}
\item HOST - przesłanie nazwy hosta
\item ABS PATH - ścieżka do pliku na serwerze
\item CONTENT LENGTH - przesłanie długości pliku
\item CREATE FILE - utworzenie dodatkowego pliku z danymi
\item RESPONSE RESULT - rezultat działania sprawdzania odpowiedzi
\item REQUEST RESULT - rezultat działania sprawdzania żądania
\end{itemize}

\section{Interfejs użytkownika}

Moduł interfejsu będzie opierał się na trybie tekstowym. Użytkownik jako argument wywołania programu może podać ścieżkę z katalogiem plików, które zostaną przefiltrowane i zostaje informowany o postępach programu.
Domyślne ustawienia programu użytkownik może zmieniać, edytując plik httpparser.conf

\chapter{Sposób użycia}

\section{Polecenia}

\subsection{Uruchomienie programu}

\begin{lstlisting}

 ./httpparser <commands>
  
  <commands> : 
     -d <directory_path>
     -l <log_file_name>
     -r - filtrowanie wszystkich możliwych danych
     --host - filtrowanie hostów
     --content-length - filtrowanie content-length
     --url - filtrowanie url
     
\end{lstlisting}

\section{Plik ustawień httpparser.conf}

\begin{lstlisting}

 # DEFAULT DIRECTORY
 ~/httpparser/files
 # DEFAULT LOG DIRECTORY
 ~/httpparser/log
 # DEFAULT LOG FILE
 ~/httpparser/log/default.log
 # DEFAULT STAT LOG FILE
 ~/httpparser/log/default_stat.log
 # DEFAULT OUTPUT FILES DIRECTORY
 ~/httpparser/output
 
 # FILTER OPTIONS
 HOST # przesłanie nazwy hosta
 ABS_PATH # ścieżka do pliku na serwerze
 CONTENT_LENGTH # przesłanie długości pliku
 CREATE_FILE # utworzenie dodatkowego pliku z danymi
 RESPONSE_ERESULT # rezultat działania sprawdzania odpowiedzi
 REQUEST_RESULT # rezultat działania sprawdzania żądania
 URL # adres url
  
\end{lstlisting}

\section{Priorytet ustawień}

\begin{enumerate}

\item Ustawienia z linii poleceń
\item Ustawienia z pliku httpparser.conf

\end{enumerate}

\section{Plik statystyk}

\subsection{Format danych}

\begin{lstlisting}  

 // dla każdego znalezionego pliku z żądaniem HTTP, w jednej linii 
 nazwa_pliku_z_żądaniem_HTTP ";" metoda ";" URL ";" 
 wielkość_pliku ";" nazwa_pliku_z_przesyłaną treścią <CRLF>
 
\end{lstlisting}

\subsection{Nazwa pliku}

\begin{lstlisting}  

 // domyślnie
 default_stat.log
 
\end{lstlisting}

\section{Pliki danych wyjściowych}

\subsection{Format danych}

Ponieważ wygenerowane dane mogą być zarówno w formacie tekstowym, jak i binarnym, będę zapisywał je w dwóch rodzajach plików.

\begin{itemize}

\item .txt - pliki tekstowe
\item .bin - pliki binarne 

\end{itemize}

\subsection{Nazwa pliku - schemat}

\begin{lstlisting}

5_cyfrowy_numer_pliku + "_" + nazwa_pliku_z_żądaniem + ".bin"
5_cyfrowy_numer_pliku + "_" + nazwa_pliku_z_żądaniem + ".txt"

\end{lstlisting}

\subsection{Przykładowe nazwy plików wyjściowych}

\begin{lstlisting}

00000_023.061.248.096.00443-192.168.000.011.56132.bin
00001_023.061.248.090.00443-192.168.000.011.54796.bin
00002_023.061.248.090.00443-192.168.000.011.54792.txt

\end{lstlisting}

\end{document}
